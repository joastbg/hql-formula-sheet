%%%%%%%%%%%%%%%%%%%%%%%%%%%%%%%%%%%%%%%%%%%%%%%%%%%%%%%%%%%%%%%%%%%%%%%%%%%%%%
% Detta �r ett exempel p� ett latexdokument.
% 
% Alla dokument best�r av f�ljande delar:
%
%          \documentclass[optioner]{dokumentklass}
%            ...inst�llningar...
%          \begin{document}
%            ...text...
%          \end{document}
%
% Som ni kanske redan har f�rst�tt �r anv�nds procent (%) f�r
% kommentarer.
%%%%%%%%%%%%%%%%%%%%%%%%%%%%%%%%%%%%%%%%%%%%%%%%%%%%%%%%%%%%%%%%%%%%%%%%%%%%%%

\documentclass[a4paper]{article}

\usepackage[T1]{fontenc}                % F�r svenska bokst�ver
\usepackage{amsmath}
\usepackage{amsfonts}
\usepackage{mathtools}
\usepackage[thmmarks,amsmath,amsthm,hyperref]{ntheorem}

%\usepackage[swedish]{babel}             % F�r svensk avstavning och svenska
                                        % rubriker (t ex "inneh�llsf�rteckning)
\title{HQL Formula Sheet}
%\date{}

\begin{document}

\maketitle                      % Skriver ut rubriken som vi
                                % deklarerade ovan med \title, \author
                                % och eventuellt \date

\section{Actual Day Count}                                
Actual/Actual Day Count (Non-leap year) 
\[ number of days in period = 365 \]
Actual/Actual Day Count (Leap year) 
\[ number of days in period = 366 \]
\section{Time value of money}

Future value
\[ FV = PV(1+r)^n \]
Present value
\[ PV = \frac{FV}{(1+r)^n} \]
Compound factor
\[ CF = (1+r)^n \]
Discount factor
\[ DF = \frac{1}{(1+r)^n} = \frac{1}{CF} \]

\subsection{Compound Interest}
\[ R = (1+r)^n \]
\subsection{Periodic Compounding}
Annual compound interest for periodic compounding, $r$ = annual nominal interest rate, $n$ = number of compoundings per year, $t$ = number of years
\[ R = (1+\frac{r}{n})^{nt} \implies (1+\frac{r}{n})^{\lfloor{nt}\rfloor} \]

\subsection{Continous Compounding}
\[ FV = V \cdot e^{rt} \]
Continous compound factor
\[ CF = e^{rt} \]

\section{Bonds}
Face value - Every bond has a fixed face value. This face value serves as the basis for the interest payment.
\subsection{Zero Coupon Bonds}
A zero coupon bond has no interim coupons and there are no interim repayments. The full amount is paid back at maturity.
F = face value, r = yield rate, t = time to maturity, Nom = nominal (100)
\[ PV = \frac{F}{(1+r)^t} \]

\subsection{Annuity}
Annuity has a fixed rate coupon, the cash flow at each settlement date remains constant during the lifetime of the bond.
Interest paid $m$ times per year $k$ periods
\[ \sum_{k=1}^{m}\frac{Nom}{(1+r)^k} \]

\subsection{Bullet}
Bullet has a fixed rate coupon which is paid at every settlement. No payments before maturity.
C = coupon, r = yield rate, n = ongoing years, N = total number of years, Nom = nominal (100)
\[ PV = \sum_{k=1}^{N}\frac{C}{(1+r)^k} + \frac{Nom}{(1+r)^N} \]

\subsection{Consol}
Consol has a fixed rate coupon. Consol bond never terminates and there is no payments at any settlement, only interest is paid.
C = coupon, r = yield rate
\[ PV = \frac{C}{r} \]

\subsection{Serial}
Serial has a fixed rate coupon, the repayments are spread evenly over all remaining settlements and the coupon payments, $C_k$ decline
over time.
\[ PV = \sum_{k=1}^{N}\frac{C_k}{(1+r)^k} + \frac{Nom}{(1+r)^N} \]

\end{document}                 % The input file ends with this command.


